
\begin{defproblem}{conjuntos1}%
 \begin{onlyproblem}%
	Confeccione un diagrama de Venn para la situación caracterizada por las siguientes condiciones. Salvo que se especifique lo contrario, suponga que todos los conjuntos tienen intersección no vacía y que ningún conjunto está incluido dentro de otro.
	
	\begin{enumerate}
		\item $ A \neq \mathcal{U} $
		\item $ B \subset C $
		\item $ A = C \cup D $
		\item $ C \cap D = \emptyset $
	\end{enumerate}

	Utilice las condiciones anteriores y las propiedades de los conjuntos para demostrar que en estas condiciones $ A^c \cup D \cup (B^c \cap (C^c \cup B))^c = \mathcal{U} $
 \end{onlyproblem}%
 \begin{onlysolution}%
 	Diagrama de Venn:
 	
 	\begin{figure}
 		\includegraphics[width=0.7\linewidth]{conjuntos1}
 		\caption[Solución a conjuntos1]{Solución a conjuntos1}
 		\label{fig:conjuntos1-solucion}
 	\end{figure}
 	
 	\begin{align*}
 	A^c \cup D \cup (B^c \cap (C^c \cup B))^c & = A^c \cup D \cup [(B^c \cap C^c) \cup (B^c \cap B)]^c &  \text{distributiva} \\	
 	& = A^c \cup D \cup [(B^c \cap C^c) \cup \emptyset]^c & \text{complementación} \\
 	& = A^c \cup D \cup [(B^c \cap C^c)]^c & \text{elemento neutro } \cup \\
 	& = A^c \cup D \cup [(B \cup C)] & \text{ley de De Morgan} \\
 	& = A^c \cup D \cup B \cup C & \text{asociativa } \cup \\
 	& = A^c \cup D \cup C \cup B & \text{conmutativa } \cup \\
 	& = A^c \cup (D \cup C) \cup B & \text{asociativa } \cup \\
 	& = A^c \cup A \cup B & \text{condición }3 \\
 	& = \mathcal{U} \cup B & \text{complementación} \\
 	& = \mathcal{U} & \text{elemento absorbente } \cup
 	\end{align*}
 \end{onlysolution}%
\end{defproblem}

\begin{defproblem}{conjuntos2}%
	\begin{onlyproblem}%
		Confeccione un diagrama de Venn para la situación caracterizada por las siguientes condiciones. Salvo que se especifique lo contrario, suponga que todos los conjuntos tienen intersección no vacía y que ningún conjunto está incluido dentro de otro.
		
		\begin{enumerate}
			\item $ A \neq \mathcal{U} $
			\item $ D \subset B $
			\item $ A = B \cup C $
			\item $ C \cap B = \emptyset $
		\end{enumerate}
		
		Utilice las condiciones anteriores y las propiedades de los conjuntos para demostrar que en estas condiciones $ A^c \cup C \cup (D^c \cap (B^c \cup D))^c = \mathcal{U} $
	\end{onlyproblem}%
	\begin{onlysolution}%
		Diagrama de Venn:
		
		\begin{figure}
			\includegraphics[width=0.7\linewidth]{conjuntos1}
			\caption[Solución a conjuntos2]{Solución a conjuntos2}
			\label{fig:conjuntos2-solucion}
		\end{figure}
	
		\begin{align*}
		A^c \cup C \cup (D^c \cap (B^c \cup D))^c & = A^c \cup C \cup [(D^c \cap B^c) \cup (D^c \cap D)]^c &  \text{distributiva} \\	
		& = A^c \cup C \cup [(D^c \cap B^c) \cup \emptyset]^c & \text{complementación} \\
		& = A^c \cup C \cup [(D^c \cap B^c)]^c & \text{elemento neutro } \cup \\
		& = A^c \cup C \cup [(D \cup B)] & \text{ley de De Morgan} \\
		& = A^c \cup C \cup D \cup B & \text{asociativa } \cup \\
		& = A^c \cup C \cup B \cup D & \text{conmutativa } \cup \\
		& = A^c \cup (C \cup B) \cup D & \text{asociativa } \cup \\
		& = A^c \cup A \cup D & \text{condición }3 \\
		& = \mathcal{U} \cup D & \text{complementación} \\
		& = \mathcal{U} & \text{elemento absorbente } \cup
		\end{align*}
	\end{onlysolution}%
\end{defproblem}

\begin{defproblem}{conjuntos3}%
	\begin{onlyproblem}%
		Confeccione un diagrama de Venn para la situación caracterizada por las siguientes condiciones. Salvo que se especifique lo contrario, suponga que todos los conjuntos tienen intersección no vacía y que ningún conjunto está incluido dentro de otro.
		
		\begin{enumerate}
			\item $ A \neq \mathcal{U} $
			\item $ B \subset A $
			\item $ A = C \cup D $
			\item $ C \cap D = \emptyset $
		\end{enumerate}
		
		Utilice las condiciones anteriores y las propiedades de los conjuntos para demostrar que en estas condiciones $ C^c \triangle D^c = A $
	\end{onlyproblem}%
	\begin{onlysolution}%
		\begin{figure}
			\includegraphics[width=0.7\linewidth]{conjuntos1}
			\caption[Solución a conjuntos3]{Solución a conjuntos3}
			\label{fig:conjuntos3-solucion}
		\end{figure}
	
		\begin{align*}
		C^c \triangle D^c & = (C^c \cup D^c) - (C^c \cap D^c) &  \text{equivalencia } \triangle \\	
		& = (C^c \cup D^c) \cap (C^c \cap D^c)^c & \text{equivalencia } - \\
		& = (C^c \cup D^c) \cap (C \cup D) & \text{Ley de De Morgan} \\
		& = (C^c \cup D^c) \cap A & \text{condición }3 \\
		& = (C \cap D)^c \cap A & \text{ley de De Morgan} \\
		& = (\emptyset)^c \cap A & \text{condición }4 \\
		& = \mathcal{U} \cap A & \text{complemento de }\emptyset \\
		& = A & \text{elemento neutro }\cap
		\end{align*}
	\end{onlysolution}%
\end{defproblem}

\begin{defproblem}{conjuntos4}%
	\begin{onlyproblem}%
		Confeccione un diagrama de Venn para la situación caracterizada por las siguientes condiciones. Salvo que se especifique lo contrario, suponga que todos los conjuntos tienen intersección no vacía y que ningún conjunto está incluido dentro de otro.
		
		\begin{enumerate}
			\item $ A \neq \mathcal{U} $
			\item $ D \subset A $
			\item $ A = C \cup B $
			\item $ C \cap B = \emptyset $
		\end{enumerate}
		
		Utilice las condiciones anteriores y las propiedades de los conjuntos para demostrar que en estas condiciones $ ((A^c \cup D) \cup C^c)^c \cap (B \cup D) = \emptyset $
	\end{onlyproblem}%
	\begin{onlysolution}%
		
		\begin{figure}
			\includegraphics[width=0.7\linewidth]{conjuntos1}
			\caption[Solución a conjuntos4]{Solución a conjuntos4}
			\label{fig:conjuntos4-solucion}
		\end{figure}
	
		\begin{align*}
		((A^c \cup D) \cup C^c)^c \cap (B \cup D) & = ((A^c \cup D)^c \cap C) \cap (B \cup D) & \text{Ley de De Morgan y doble complemento} \\	
		& = ((A \cap D^c) \cap C) \cap (B \cup D) & \text{Ley de De Morgan y doble complemento} \\
		& = A \cap D^c \cap C \cap (B \cup D) & \text{asociativa } \cap \\
		& = A \cap D^c \cap [(C \cap B) \cup (C \cap D)] & \text{distributiva } \\
		& = A \cap D^c \cap [(\emptyset) \cup (C \cap D)] & \text{condición }4 \\
		& = A \cap D^c \cap (C \cap D) & \text{elemento neutro }\cup \\
		& = A \cap D^c \cap C \cap D & \text{asociativa }\cup \\
		& = A \cap (D^c \cap D) \cap C & \text{conmutativa y asociativa }\cup \\
		& = A \cap \emptyset \cap C & \text{complementación} \\
		& = \emptyset & \text{elemento absorbente }\cap
		\end{align*}
	\end{onlysolution}%
\end{defproblem}