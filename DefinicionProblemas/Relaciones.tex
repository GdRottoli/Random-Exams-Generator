
\begin{defproblem}{relaciones1}%
 \begin{onlyproblem}%
	En los juegos olímpicos desarrollados cada cuatro años compiten países de todos los continentes. El desempeño de cada país está medido por la cantidad de medallas de oro (G), plata (P) y bronce (B) que obtienen.
	
	El ranking de países para una edición determinada se construye ordenando los países según el siguiente predicado:
	
	\[ Pais_1 \succeq Pais2 \Leftrightarrow (G_1 \ge G_2) \land (G_1 + P_1) \ge (G_2 + P_2) \land (G_1+P_1+B_1) \ge (G_2+P_2+B_2) \]
	
	siendo $ G_i $ la cantidad de medallas de oro, $ P_i $ la cantidad de medallas de plata y $ B_i $ la cantidad de medallas de bronce obtenidas por el país $ i $ en una misma edición.
	
	En Atenas 2004 se obtuvieron estos 6 primeros puestos: 
	
	\begin{tabular}{|lccc|}
		\hline
		País & G & P & B \\
		\hline
		Alemania (GER)  & 13 & 16 & 20 \\
		Australia (AUS)  & 17 & 16 & 16 \\
		China (CHN) & 32 & 17 & 14 \\
		Estados Unidos (USA)  & 36 & 39 & 26 \\
		Japón (JPN)  & 16 & 9 & 12 \\
		Rusia (RUS)  & 28 & 27 & 38 \\
		\hline
	\end{tabular}
	
	\begin{enumerate}
		\item Demuestre que la relación R definida por el predicado $ x R y \Leftrightarrow x \text{ e } y \text{ son del mismo continente} $ es una relación de equivalencia.
		\item Identifique el conjunto cociente de la relación anterior.
		\item Demuestre que $ \succeq $ es una relación de orden.
		\item Construya el diagrama de Hasse del conjunto ordenado correspondiente. ¿Está total o parcialmente ordenado?
		\item Identifique maximales y minimales. ¿Hay máximo? ¿Hay mínimo?
	\end{enumerate}
	
 \end{onlyproblem}%
 \begin{onlysolution}%
 
 \end{onlysolution}%
\end{defproblem}

\begin{defproblem}{relaciones2}%
	\begin{onlyproblem}%
		En los juegos olímpicos desarrollados cada cuatro años compiten países de todos los continentes. El desempeño de cada país está medido por la cantidad de medallas de oro (G), plata (P) y bronce (B) que obtienen.
		
		El ranking de países para una edición determinada se construye ordenando los países según el siguiente predicado:
		
		\[ Pais_1 \succeq Pais2 \Leftrightarrow (G_1 \ge G_2) \land (G_1 + P_1) \ge (G_2 + P_2) \land (G_1+P_1+B_1) \ge (G_2+P_2+B_2) \]
		
		siendo $ G_i $ la cantidad de medallas de oro, $ P_i $ la cantidad de medallas de plata y $ B_i $ la cantidad de medallas de bronce obtenidas por el país $ i $ en una misma edición.
		
		En Pekin 2008 se obtuvieron estos 6 primeros puestos: 
		
		\begin{tabular}{|lccc|}
			\hline
			País & G & P & B \\
			\hline
			Alemania (GER)  & 16 & 11 & 14 \\
			Reino Unido (GBR)  & 19 & 13 & 17 \\
			Australia (AUS)  & 14 & 15 & 17 \\
			Rusia (RUS)  & 24 & 13 & 23 \\
			China (CHN)  & 48 & 22 & 30 \\
			Estados Unidos (USA)  & 36 & 39 & 37 \\
			\hline
		\end{tabular}
	
		\begin{enumerate}
			\item Demuestre que la relación R definida por el predicado $ x R y \Leftrightarrow x \text{ e } y \text{ son del mismo continente} $ es una relación de equivalencia.
			\item Identifique el conjunto cociente de la relación anterior.
			\item Demuestre que $ \succeq $ es una relación de orden.
			\item Construya el diagrama de Hasse del conjunto ordenado correspondiente. ¿Está total o parcialmente ordenado?
			\item Identifique maximales y minimales. ¿Hay máximo? ¿Hay mínimo?
		\end{enumerate}
		
	\end{onlyproblem}%
	\begin{onlysolution}%
		
	\end{onlysolution}%
\end{defproblem}

\begin{defproblem}{relaciones3}%
	\begin{onlyproblem}%
		En los juegos olímpicos desarrollados cada cuatro años compiten países de todos los continentes. El desempeño de cada país está medido por la cantidad de medallas de oro (G), plata (P) y bronce (B) que obtienen.
		
		El ranking de países para una edición determinada se construye ordenando los países según el siguiente predicado:
		
		\[ Pais_1 \succeq Pais2 \Leftrightarrow (G_1 \ge G_2) \land (G_1 + P_1) \ge (G_2 + P_2) \land (G_1+P_1+B_1) \ge (G_2+P_2+B_2) \]
		
		siendo $ G_i $ la cantidad de medallas de oro, $ P_i $ la cantidad de medallas de plata y $ B_i $ la cantidad de medallas de bronce obtenidas por el país $ i $ en una misma edición.
		
		En Londres 2012 se obtuvieron estos 6 primeros puestos: 
		
		\begin{tabular}{|lccc|}
			\hline
			País & G & P & B \\
			\hline
			Alemania (GER)  & 11 & 19 & 14 \\
			China (CHN)  & 38 & 31 & 22 \\
			Corea del Sur (KOR)  & 13 & 8 & 7 \\
			Estados Unidos (EE UU)  & 46 & 28 & 29 \\
			Reino Unido (GBR)  & 29 & 17 & 19 \\
			Rusia (RUS)  & 24 & 26 & 32 \\
			\hline
		\end{tabular}
	
		\begin{enumerate}
			\item Demuestre que la relación R definida por el predicado $ x R y \Leftrightarrow x \text{ e } y \text{ son del mismo continente} $ es una relación de equivalencia.
			\item Identifique el conjunto cociente de la relación anterior.
			\item Demuestre que $ \succeq $ es una relación de orden.
			\item Construya el diagrama de Hasse del conjunto ordenado correspondiente. ¿Está total o parcialmente ordenado?
			\item Identifique maximales y minimales. ¿Hay máximo? ¿Hay mínimo?
		\end{enumerate}
		
	\end{onlyproblem}%
	\begin{onlysolution}%
		
	\end{onlysolution}%
\end{defproblem}

\begin{defproblem}{relaciones4}%
	\begin{onlyproblem}%
		En los juegos olímpicos desarrollados cada cuatro años compiten países de todos los continentes. El desempeño de cada país está medido por la cantidad de medallas de oro (G), plata (P) y bronce (B) que obtienen.
		
		El ranking de países para una edición determinada se construye ordenando los países según el siguiente predicado:
		
		\[ Pais_1 \succeq Pais2 \Leftrightarrow (G_1 \ge G_2) \land (G_1 + P_1) \ge (G_2 + P_2) \land (G_1+P_1+B_1) \ge (G_2+P_2+B_2) \]
		
		siendo $ G_i $ la cantidad de medallas de oro, $ P_i $ la cantidad de medallas de plata y $ B_i $ la cantidad de medallas de bronce obtenidas por el país $ i $ en una misma edición.
		
		En Río de Janeiro 2016 se obtuvieron estos 6 primeros puestos: 
		
		\begin{tabular}{|lccc|}
			\hline
			País & G & P & B \\
			\hline
			Alemania (GER)  & 17 & 10 & 15 \\
			China (CHN)  & 26 & 18 & 26 \\
			Estados Unidos (USA)  & 46 & 37 & 38 \\
			Japón (JPN)  & 12 & 8 & 21 \\
			Reino Unido (GBR)  & 27 & 23 & 17 \\
			Rusia (RUS)  & 19 & 18 & 19 \\
			\hline
		\end{tabular}
		
		\begin{enumerate}
			\item Demuestre que la relación R definida por el predicado $ x R y \Leftrightarrow x \text{ e } y \text{ son del mismo continente} $ es una relación de equivalencia.
			\item Identifique el conjunto cociente de la relación anterior.
			\item Demuestre que $ \succeq $ es una relación de orden.
			\item Construya el diagrama de Hasse del conjunto ordenado correspondiente. ¿Está total o parcialmente ordenado?
			\item Identifique maximales y minimales. ¿Hay máximo? ¿Hay mínimo?
		\end{enumerate}
		
	\end{onlyproblem}%
	\begin{onlysolution}%
		
	\end{onlysolution}%
\end{defproblem}