% This is an example file accompanying "LaTeX for Administrative Work"
% and was adapted from the file prob-1stprncp.tex in the probsoln.sty bundle.
%
% These problems all involve differentiating from 1st principles

\begin{defproblem}{lpc:mosqueteros1}%
 \begin{onlyproblem}%
Los Mosqueteros del Rey se percataron de la presencia de una persona enmascarada en el parque del palacio. 
D'Artagnan cree que era el Cardenal. 
Portos dijo que el desconocido es bien el Cardenal o bien Milady. 
Aramis afirma que si Portos no está equivocado, entonces D'Artagnan está en lo cierto. 
Atos está seguro de que Portos y Aramis no pueden estar ambos errados. 
Investigaciones posteriores demostraron que las conclusiones de Atos eran correctas. 
\begin{enumerate}
	\item Utilice Lógica Proposicional para escribir simbólicamente estos enunciados.
	\item Mediante razonamiento directo o indirecto responda: ¿quién estaba bajo la máscara?
\end{enumerate}
 \end{onlyproblem}%
 \begin{onlysolution}%
 
 \end{onlysolution}%
\end{defproblem}

\begin{defproblem}{lpc:mosqueteros2}%
 \begin{onlyproblem}%
 El capitán de los mosqueteros del rey, De Treville, ha recibido tres cartas de sus subordinados.
 La primera afirma que si Atos ha roto su espada en batalla, entonces Portos y Aramis han roto también las suyas.
 La segunda informa que Atos y Aramis han roto ambos sus espadas o ambos las han preservado.
 Finalmente, la tercera afirma que para que Aramis rompa su espada es necesario que Portos rompa la suya.
 De Treville además sabe que una de las tres cartas enviadas por los mosqueteros ha sido interceptada y contiene información falsa.
 \begin{enumerate}
 	\item Utilice Lógica Proposicional para escribir simbólicamente estos enunciados.%
 	\item Mediante razonamiento directo o indirecto responda: ¿quién rompió su espada?%
 \end{enumerate}
 \end{onlyproblem}%
 \begin{onlysolution}%

 \end{onlysolution}%
\end{defproblem}

\begin{defproblem}{lpc:jobshop1}%
 \begin{onlyproblem}%
 Un flujo de trabajo provee el siguiente esquema de operación de cuatro máquinas $ S_1-S_4 $. 
 Si la primera máquina está trabajando, entonces la segunda y la tercera también trabajan.
 La tercera máquina trabaja cuando y sólo cuando la cuarta está trabajando.
 Aparte de esto, si la segunda está trabajando, la cuarta debe estar detenida.
 \begin{enumerate}
 	\item Utilice Lógica Proposicional para escribir simbólicamente estos enunciados.%
 	\item Mediante razonamiento directo o indirecto encuentre cuáles de las máquinas están trabajando si se sabe que la primera o la segunda máquina están trabajando (pero no simultáneamente)%
 \end{enumerate}
 
 \end{onlyproblem}%
 \begin{onlysolution}%
 
 \end{onlysolution}
\end{defproblem}

\begin{defproblem}{lpc:jobshop2}%
 \begin{onlyproblem}%
 Un flujo de trabajo provee el siguiente esquema de operación de cuatro máquinas $ S_1-S_4 $. 
 Si la primera máquina está trabajando, entonces la segunda también.
 La segunda máquina trabaja cuando y sólo cuando la tercera está trabajando.
 Además de esto, si la cuarta máquina está trabajando, la tercera debe estar detenida.
 \begin{enumerate}
 	\item Utilice Lógica Proposicional para escribir simbólicamente estos enunciados.%
 	\item Mediante razonamiento directo o indirecto encuentre cuáles de las máquinas están trabajando si se sabe que la primera o la segunda máquina están trabajando (pero no simultáneamente)%
 \end{enumerate}
 
 \end{onlyproblem}%
 \begin{onlysolution}%
 
 \end{onlysolution}%
\end{defproblem}

\begin{defproblem}{lpc:estudiantes1}%
 \begin{onlyproblem}%
 Cuatro estudiantes están de vacaciones en un centro recreativo: Alicia, Bartolomé, Valeria y Jorge. Algunos de ellos están de excursión, mientras que otros están tomando sol en la playa.
 Se sabe que Bartolomé o Valeria están de excursión (tal vez ambos).
 Si Valeria y Jorge están en la playa, entonces Alicia también.
 No es cierto que si Alicia está en la playa entonces Valeria y Jorge están de excursión.
 \begin{enumerate}
 	\item Utilice Lógica Proposicional para escribir simbólicamente estos enunciados.%
 	\item Mediante razonamiento directo o indirecto responda: ¿Cuáles de los estudiantes están de viaje y cuáles en la playa?%
 \end{enumerate}
 
 \end{onlyproblem}%
 \begin{onlysolution}%
 
 \end{onlysolution}%
\end{defproblem}

\begin{defproblem}{lpc:estudiantes2}%
	\begin{onlyproblem}%
		Cuatro estudiantes están de vacaciones en un campamento: Alicia, Bartolomé, Valeria y Jorge. Algunos de ellos hacen deporte, mientras que otros se preparan para un espectáculo amateur.
		Se sabe que Alicia o Bartolomé hacen deporte (tal vez ambos).
		Si Bartolomé y Valeria están preparándose para el espectáculo, entonces Jorge también.
		Más aún, no es cierto que si Jorge se está preparando para el espectáculo entonces Bartolomé y Valeria hacen deporte.
		
		\begin{enumerate}
			\item Utilice Lógica Proposicional para escribir simbólicamente estos enunciados.%
			\item Mediante razonamiento directo o indirecto responda: ¿Cuáles de los estudiantes hacen deporte y cuáles se preparan para el espectáculo?%
		\end{enumerate}
		
	\end{onlyproblem}%
	\begin{onlysolution}%
		
	\end{onlysolution}%
\end{defproblem}